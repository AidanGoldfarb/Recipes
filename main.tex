\documentclass[10pt]{article}

\usepackage{amsmath}
\usepackage{hyperref}
\usepackage{tikz-cd}
\usepackage{amssymb}
\usepackage{amsthm}
\usepackage{bm}
\usepackage{listings}
\usepackage{bbm}
\usepackage{multicol}
\usepackage{mathtools}
\usepackage{mathpartir}
\usepackage{float}
\usepackage{enumerate}
\usepackage[margin=1.25in]{geometry}
\usepackage[T1]{fontenc}
\usepackage{kpfonts}

\input{../Reed/School/LaTeX/macros.tex}
\input{../Reed/School/LaTeX/lemmas.tex}

\begin{document}

\subsection*{Soups, Stews, and the Like}

\subsubsection*{The basic method for stewing}

This works well for several kinds of meat, including beef and pork (the author's preferred meats, rather than, say, chicken).

Take a small bowl of flour and add salt, then flour meat, which has been trimmed as you please and cut into bite-sized pieces.
Brown the meat in the same pot you intend to cook the stew in, with a generous helping of butter in the pot.
You may use vegetable oil instead, but it will be worse.
Remove the meat once browned on two sides and set aside.

Add an onion (and celery, if desired/it is to be had) and cook until translucent and soft; don't let it brown.
Add a generous amount of garlic and cook for a minute or two.
Once this is done, add whatever herbs and spices you intend to use---for an herby flavor, oregano, thyme, and rosemary work well.
Almost every stew benefits from a couple of bay leaves.
Let this cook a minute, then add the wine.
The type of wine depends on the meat: pair as you usually would.

Once the wine has reduced a bit, add the meat and the broth, and bring to a boil.
Now is the time to skim fat from the top of the stew.
Simmer for about an hour.
Then add whatever vegetables you please (potatoes and carrots are always a welcome addition), and simmer for another half hour, or until the vegetables are done to your liking.

\subsubsection*{To make chili the Texas way}

This mostly follows the basic method for stewing as described earlier.

Trim some of the fat off of a 2 lbs. chuck roast and cut into slightly-large-than-bite-sized pieces.
Salt and pepper the meat, then brown in the pot with a generous amount of butter.
This may take multiple batches, and simply set aside the meat when it is browned on two sides.
Reserve some for eating while you cook!

Once the meat is browned, add very roughly cut onions and green bell peppers (about one large onion and two medium bell peppers).
Cook until the onions are soft and translucent, then add chopped jalape\~{n}os (about three for a pleasant but noticeable spiciness) with seeds removed.
Cook for a couple more minutes, then add garlic, and couple for another minute or two.
Then add the chili powder, made from about 8 large chilis (a mixture of pasilla, chipotle, and ancho work well).
Also add a couple tablespoons of paprika and cumin, and about half a tablespoon of coriander powder.
Mix and cook for a minute to let the spices cook.

Then add the beer and beef broth, and the meat that was set aside earlier, along with all of its juices.
Bring to a boil, skimming the fat off the top.
Then simmer for about 3 hours.

Serve with chips, sour cream, and cheese.

\subsection*{Food for Breakfast}

You can eat any food any time (you can send anyone who disagrees to me), but these are the traditional ``breakfast foods.''

\subsubsection*{To make 5 Pancakes}

To 1/2 cup flour put 2/5 tsp of salt, 1 tsp of baking powder, and some sugar.
Mix.
Add an egg, 1/4 cup of milk, a tablespoon of butter and vanilla (if you please).
Mix.
Spoon into equal portions and cook over medium heat in a lightly oiled pan.

\subsubsection*{How to make buttered eggs}

This recipe is 90\% technique, and 10\% ingredients.
It works best for 1-3 eggs, and if you wish to make more, it is best done in multiple batches.
But your eggs are your eggs and you may do as you please.
The goal is to make eggs that have a just-barely-firm outside and a slightly runny and creamy inside.

Take a tablespoon of butter per egg and melt in a pan which only so large as to have the eggs form a thin layer over the entire pan.
Spread butter evenly across pan as it melts, and when it is melted, remove from the heat, and add the eggs.
Mix the eggs with a spatula after adding a pinch of salt (you may do this in a bowl first, but this takes fewer dishes).
Take care not to mix too well---there should be visibly separate bits of white and yolk still.
Return the pan to the heat, and let sit briefly until the eggs start to solidify on the bottom.
Then, using the spatula, scrape the eggs from the bottom of the pan frequently, never letting them sit for too long.
Fold the eggs in on themselves so they begin to come together into a coherent lump.
Once they have just done so, turn off the heat, and continue folding until the bottom of the eggs has just become firm.
Then flip the eggs over briefly, and serve immediately.

This recipe is only exceptional so long as the eggs are warm.

\subsubsection*{Frying potatoes}

Take about one medium potato per person.
These should be precooked, either baked in the oven as a baked potato (leftover baked potatoes work well for this), or in the microwave for about 3 minutes per side (add 15-30 seconds per potato, including the first).
If done in the microwave, poke with a fork before microwaving.

Then cut the potatoes into small cubes, and put into a pan with a generous layer of a mixture of melted butter and olive oil; the pan should be large enough to leave some space between the potatoes.
Season with salt and pepper.
Then cook, taking care not to stir too often so that a good crust can form on the potatoes.
If you wish, you may add onions (about half an onion per potato), garlic (lots), or both several minutes after the potatoes have begun frying, as they will burn otherwise.
Near the end, you may also add paprika.

Once potatoes are well-browned on multiple sides, remove from heat and serve.

\subsubsection*{Applesauce}

You will need a vessel with a lid that can be microwaved, and as many apples, of whatever variety you please, as will fit, unchopped, into the vessel.
This recipe assumes your vessel fits roughly ten medium apples.
Peel and roughly chop the apples.
Add a small amount of water to the vessel.
Then add a layer of apples, cover with a generous amount of sugar and cinnamon, and a small amount of nutmeg.
Continue in this way until the vessel is full, and add the juice of one lemon.

Then, place vessel in microwave and cook in four to six minute increments, stirring between each.
Once the apples are softened to your liking, mash them, and serve it forth.

\subsection*{Baked Goods}

\end{document}

